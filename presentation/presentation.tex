\documentclass{beamer}
\title{Understanding The Haskell FFI}
\author{Tim Skinner \\ {\small tim@timskinner.net}}
\date{\today}
\mode<presentation> { \usetheme{Berlin} }

\usepackage[english]{babel}
\usepackage[latin1]{inputenc}
\usepackage{times}
\usepackage[T1]{fontenc}

\AtBeginSubsection[]
{
    \begin{frame}<beamer>{Outline}
        \tableofcontents[currentsection,currentsubsection]
    \end{frame}
}

\begin{document}
\begin{frame}
    \titlepage
\end{frame}

\section*{Outline}
\begin{frame}
    \begin{tiny}
        \tableofcontents
    \end{tiny}
\end{frame}

\section{Introduction}
\subsection{What is the FFI?}
\subsection{About the examples}

\section{Data Types}
\subsection{Foreign Types}
\subsection{Haskell Types}
\subsection{The Storable Typeclass}

\section{Functions}
\subsection{Calling Native Functions}
\subsection{Exposing Haskell Functions}
\subsection{Purity}

\section{Tools}
\subsection{Compilation and Linking}
\subsection{Automated Wrappers}

\section{Idioms}

\begin{frame}
\end{frame}

\end{document}
