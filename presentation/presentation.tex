\documentclass{beamer}
\title{Understanding The Haskell FFI}
\author{Tim Skinner \\ {\small tim@timskinner.net}}
\date{\today}
\mode<presentation> { \usetheme{Berlin} }

\usepackage[english]{babel}
\usepackage[latin1]{inputenc}
\usepackage{times}
\usepackage[T1]{fontenc}

\AtBeginSubsection[]
{
    \begin{frame}<beamer>{Outline}
        \tableofcontents[currentsection,currentsubsection]
    \end{frame}
}

\begin{document}
\begin{frame}
    \titlepage
\end{frame}

\section*{Outline}
\begin{frame}
    \begin{tiny}
        \tableofcontents
    \end{tiny}
\end{frame}

\section{Introduction}
\subsection{What is the FFI?}
\subsection{About the examples}

\section{Linkers and Loaders}
\subsection{The ELF File Format}
\subsection{Static and Dynamic Linking}
\subsection{Calling Conventions}

\section{Data Types}
\subsection{Foreign Types}
\subsection{The Storable Typeclass}

\section{Function Calls}
\subsection{Purity and Side Effects}
\subsection{Function Pointers}

\section{Native Functions}
\subsection{Declaring Native Functions}
\subsection{Compilation and Linking}

\section{Accessing Haskell Functions from Foreign Code}
\subsection{Declaring Native-Accessible Functions}
\subsection{Compilation}
\begin{frame}
\end{frame}

\end{document}
