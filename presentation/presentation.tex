
\documentclass{beamer}
\title{Understanding The Haskell FFI}
\author{Tim Skinner}
\date{\today}
\mode<presentation> { \usetheme{Berlin} }

\usepackage[english]{babel}
\usepackage[latin1]{inputenc}
\usepackage{times}
\usepackage[T1]{fontenc}
\usepackage{hyperref}
\usepackage{listings}
\usepackage{color}

\definecolor{comment}{rgb}{0,0.6,0}
\definecolor{keyword}{rgb}{0,0,0.6}
\definecolor{string}{rgb}{0.58,0,0.82}

\lstset {
    backgroundcolor=\color{white},
    basicstyle=\tiny,
    keywordstyle=\color{keyword},
    commentstyle=\color{comment},
    showspaces=false,
    showstringspaces=false,
    showtabs=false,
    stringstyle=\color{string},
    tabsize=4
}

\newcommand{\chref}[3] {
    {\color{#1} \href{#2}{\underline{#3}}}
}

\newcommand{\cmd}[1]{
    \vspace{2mm}
    \fcolorbox{gray}{black}{\tt {\color{white} user@host\$ #1}}
    \vspace{2mm}\\
}

\begin{document}

\begin{frame}
    \titlepage
\end{frame}

\section*{Outline}
\begin{frame}
    \begin{tiny}
        \tableofcontents
    \end{tiny}
\end{frame}

\section{Introduction}
\begin{frame}
    \frametitle{What is the FFI?}

    The FFI allows Haskell code to interoperate with native code by allowing
    haskell applications to call or be called by native functions through
    static and shared libraries and object files.

\end{frame}

\begin{frame}
    \frametitle{What is Native code?}

    The \chref{darkgray}
    {http://www.cse.unsw.edu.au/~chak/haskell/ffi/ffi/ffi.html} {Haskell 98
    Addendum on FFI} defines a mechanism for interoperating with code that uses
    the platforms C calling convention.

    The standard leaves room for implementations to support other conventions,
    such as C++ or Java, but these are not supported by GHC.
\end{frame}

\begin{frame}
    \frametitle{A Brief Aside on Platform Dependence}

    Since the FFI deals with implementation defined and platform specific code,
    we will pick a reference platform for the examples.  In this case:

    \begin{itemize}
        \item {GNU + Linux}
        \item {\chref{darkgray}{http://www.uclibc.org/docs/psABI-x86_64.pdf}{AMD64 System V ABI}}
        \item {\chref{darkgray}{http://www.skyfree.org/linux/references/ELF_Format.pdf}{ELF File Format}}
        \item {GHC 7.4}
        \item {GCC 4.7}
        \item {libc 4.6}
    \end{itemize}
\end{frame}

\section{Symbols, Linking, and ABIs}
\begin{frame}
    \frametitle{Background}

    To understand how the FFI on works on our target platform we need to
    understand how C applications work.  Let's look at how we go from source
    code to a running application on our target platform.
\end{frame}

\begin{frame}
    \frametitle {Our first example - {\tt hello.c}}
    \lstinputlisting[language=C]{"../samples/hello_c/hello.c"}
\end{frame}

\begin{frame}
    \frametitle{Compiling Files}

    Although we can generate an executable directly from our source code, it's
    illustrative to first generate an object file:

    \cmd{gcc -std=gnu99 -c hello.c -o hello.o}

    Next we can link our object file with the system libraries to generate our
    final executable. gcc is helping us out here by defining some default
    parameters, but we could also do this manually by running {\tt ld}
    directly.

    \cmd{gcc hello.o -o hello}
\end{frame}

\begin{frame}
    \frametitle{The ELF Object File Format}

    The ELF file format consists of an ELF header containing metadata
    information and offets to a number of sections.  The specific sections that
    are included in a file vary depending on the type of file.  Of specific
    interest to us are the \emph{Symbol Table} and the \emph{Relocations}

    We can use the {\tt readelf} command to look at the contents of an ELF
    file.

\end{frame}

\begin{frame}[fragile]
    \frametitle{ELF Object File Symbol Table}
    \begin{tiny}
     \begin{verbatim}
Symbol table '.symtab' contains 14 entries:
   Num:    Value          Size Type    Bind   Vis      Ndx Name
     0: 0000000000000000     0 NOTYPE  LOCAL  DEFAULT  UND
     1: 0000000000000000     0 FILE    LOCAL  DEFAULT  ABS hello.c
     2: 0000000000000000     0 SECTION LOCAL  DEFAULT    1
     3: 0000000000000000     0 SECTION LOCAL  DEFAULT    3
     4: 0000000000000000     0 SECTION LOCAL  DEFAULT    4
     5: 0000000000000000     0 SECTION LOCAL  DEFAULT    5
     6: 0000000000000000     0 SECTION LOCAL  DEFAULT    7
     7: 0000000000000000     0 SECTION LOCAL  DEFAULT    8
     8: 0000000000000000     0 SECTION LOCAL  DEFAULT    6
     9: 0000000000000000    52 FUNC    GLOBAL DEFAULT    1 generate_message
    10: 0000000000000000     0 NOTYPE  GLOBAL DEFAULT  UND asprintf
    11: 0000000000000034    60 FUNC    GLOBAL DEFAULT    1 main
    12: 0000000000000000     0 NOTYPE  GLOBAL DEFAULT  UND puts
    13: 0000000000000000     0 NOTYPE  GLOBAL DEFAULT  UND free
    \end{verbatim}
    \end{tiny}
\end{frame}

\begin{frame}[fragile]
    \frametitle{ELF Object File .text Relocations}
    \begin{tiny}
        \begin{verbatim}
Relocation section '.rela.text' at offset 0x678 contains 6 entries:
  Offset          Info           Type           Sym. Value    Sym. Name + Addend
00000000001d  00050000000a R_X86_64_32       0000000000000000 .rodata + 0
00000000002a  000a00000002 R_X86_64_PC32     0000000000000000 asprintf - 4
000000000044  00050000000a R_X86_64_32       0000000000000000 .rodata + a
000000000049  000900000002 R_X86_64_PC32     0000000000000000 generate_message - 4
000000000059  000c00000002 R_X86_64_PC32     0000000000000000 puts - 4
000000000065  000d00000002 R_X86_64_PC32     0000000000000000 free - 4
        \end{verbatim}
    \end{tiny}
\end{frame}


\section{Marshalling}
% \subsection{Foreign Types}
% \subsection{Haskell Types}
% \subsection{The Storable Typeclass}

\section{Functions}
% \subsection{Calling Native Functions}
% \subsection{Exposing Haskell Functions}
% \subsection{Purity}

\section{Tools}
% \subsection{Compilation and Linking}
% \subsection{Automated Wrappers}

\section{Idioms}

\begin{frame}
\end{frame}

\end{document}
